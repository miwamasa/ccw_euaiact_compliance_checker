% !TEX program = lualatex
\documentclass[11pt,a4paper]{article}

% Packages
\usepackage{luatexja}
\usepackage{luatexja-fontspec}
\usepackage{amsmath,amssymb,amsfonts}
\usepackage{amsthm}
\usepackage{graphicx}

% Theorem-like environments
\theoremstyle{definition}
\newtheorem{definition}{Definition}[section]
\newtheorem{theorem}{Theorem}[section]
\newtheorem{example}{Example}[section]

\usepackage{booktabs}
\usepackage{algorithm}
\usepackage{algpseudocode}
\usepackage{hyperref}
\usepackage{xcolor}
\usepackage{listings}
\usepackage{tikz}
\usetikzlibrary{trees,arrows,shapes}
\usepackage{geometry}
\geometry{margin=2.5cm}

% Code listing style
\lstset{
  basicstyle=\ttfamily\small,
  keywordstyle=\color{blue},
  commentstyle=\color{gray},
  stringstyle=\color{red},
  frame=single,
  breaklines=true
}

\title{規制遵守の意思決定木に対する情報理論的最適化:\\
EU AI Act 適合性チェックのケーススタディ}

\author{%
  意思決定木最適化 研究\\
  \texttt{EU AI Act Service Desk 分析に基づく}
}

\date{2026年1月}

\begin{document}

\maketitle

\begin{abstract}
本稿では,規制遵守チェックに用いられる意思決定木(質問票)の平均質問数を削減するために,シャノンエントロピーと情報利得に基づく最適化枠組みを提案する.EU AI Act の適合性チェッカーを題材に,意思決定ロジックの同値性(法的正確性)を維持したまま,質問の並べ替え・統合・早期終了の余地を定量的に評価する.
\end{abstract}

% NOTE: 本ファイルは日本語版の雛形です。
% 本文の完全な日本語化は未実施のため,必要に応じて各節を翻訳して追記してください。

\section{はじめに}

\subsection{動機}

EU Artificial Intelligence Act(Regulation (EU) 2024/1689)は,EU域内のAIシステムに関する包括的な規制枠組みを定める.組織が自らの遵守義務を判断するには,以下のような要因に依存する複雑な意思決定木(質問票フロー)を辿る必要がある.

\begin{itemize}
    \item AIシステムの種別(汎用目的AIモデル(GPAI)かAIシステムか)
    \item 組織の役割(提供者,導入者,販売者等)
    \item リスク分類(禁止,高リスク,透明性義務等)
    \item 分野別規制(Annex I/IIIのカテゴリ等)
\end{itemize}

European Commission の AI Act Service Desk には,AI System と GPAI の2トラックからなるコンプライアンスチェッカー(全33問)が提供されている.しかし,現行フローは「網羅性」優先で設計されており,利用者が必要以上の質問に回答する状況が生じ得る.

\subsection{貢献}

本稿の貢献は以下の通りである.

\begin{enumerate}
    \item 規制遵守領域における意思決定木最適化の形式的枠組み
    \item EU AI Act 適合性チェッカーに対する情報理論的分析
    \item 7,762本の意思決定パスのプログラム的列挙と指標算出
    \item 改善案(早期終了,統合,並べ替え)の提示と効果の定量化
    \item 古典的意思決定木アルゴリズム(ID3, C4.5, CART)との比較
\end{enumerate}

\section{理論的基盤}

\subsection{問題定式化}

質問集合を $\mathcal{Q} = \{q_1, q_2, \ldots, q_n\}$,フラグ集合を $\mathcal{F} = \{f_1, f_2, \ldots, f_m\}$,義務集合を $\mathcal{O} = \{o_1, o_2, \ldots, o_k\}$ とする.

\begin{definition}[Decision Tree]
意思決定木 $T = (V, E)$ は次で構成される.
\begin{itemize}
    \item $V$: 状態(質問または終端決定)を表すノード
    \item $E$: 回答を表す有向辺
\end{itemize}
\end{definition}

\begin{definition}[State Vector]
状態 $s \in \mathcal{S}$ を
\begin{equation}
s = (\text{answers}, \text{flags}, \text{obligations})
\end{equation}
で定義する.ここで answers は部分関数 $\mathcal{Q} \rightarrow \mathcal{A}$ である.
\end{definition}

\subsection{最適化目的}

法的正確性を保ったまま期待パス長を最小化する.

\begin{equation}
\min_{T} \mathbb{E}[\text{path\_length}(T)] = \sum_{i} p(\text{path}_i) \times |\text{path}_i|
\end{equation}

制約条件は以下である.
\begin{align}
\forall s \in \mathcal{S}: & \quad \text{correct\_classification}(s) = \text{TRUE} \\
\forall s \in \mathcal{S}: & \quad \text{legal\_compliance}(s) = \text{TRUE} \\
\forall s \in \mathcal{S}: & \quad \text{complete\_coverage}(s) = \text{TRUE}
\end{align}

\subsection{情報理論の基礎}

\subsubsection{シャノンエントロピー}

状態空間の不確実性は次で測られる.

\begin{equation}
H(\mathcal{S}) = -\sum_{s \in \mathcal{S}} p(s) \log_2 p(s)
\end{equation}

\subsubsection{情報利得}

質問 $q$ の情報利得は,観測により期待されるエントロピー減少量である.

\begin{equation}
\text{IG}(q, \mathcal{S}) = H(\mathcal{S}) - \sum_{v \in \text{answers}(q)} p(v) \times H(\mathcal{S} | q = v)
\end{equation}

ここで $H(\mathcal{S} | q = v)$ は $q=v$ を観測した後の条件付きエントロピーである.

\subsubsection{最適質問選択}

各ノードで最大情報利得の質問を貪欲に選択する.

\begin{equation}
q^* = \arg\max_{q \in \mathcal{Q}_{\text{remaining}}} \text{IG}(q, \mathcal{S}_{\text{current}})
\end{equation}

\section{関連研究:古典的意思決定木アルゴリズム}

\subsection{ID3}

Quinlan (1986) による ID3 は情報利得を分割基準として用いる.

\begin{algorithm}
\caption{ID3 Algorithm}
\begin{algorithmic}[1]
\Function{ID3}{$D$, $\mathcal{Q}$}
    \If{all examples in $D$ have same class}
        \State \Return leaf node with that class
    \EndIf
    \If{$\mathcal{Q}$ is empty}
        \State \Return leaf with majority class
    \EndIf
    \State $q^* \gets \arg\max_{q \in \mathcal{Q}} \text{IG}(q, D)$
    \State Create node for $q^*$
    \For{each value $v$ of $q^*$}
        \State Add subtree \Call{ID3}{$D_v$, $\mathcal{Q} \setminus \{q^*\}$}
    \EndFor
    \State \Return tree
\EndFunction
\end{algorithmic}
\end{algorithm}

\subsection{C4.5}

C4.5 (Quinlan, 1993) は多値属性への偏りを抑えるため \textbf{gain ratio} を用いる.

\begin{equation}
\text{GainRatio}(q) = \frac{\text{IG}(q)}{\text{SplitInfo}(q)}
\end{equation}

\begin{equation}
\text{SplitInfo}(q) = -\sum_{v} \frac{|D_v|}{|D|} \log_2 \frac{|D_v|}{|D|}
\end{equation}

\subsection{CART}

CART (Breiman et al., 1984) は \textbf{Gini impurity} を用い,二分木を生成する.

\begin{equation}
\text{Gini}(D) = 1 - \sum_{k} p_k^2
\end{equation}

規制質問は多肢選択が多く,常に二分木にすることは適さない場合がある.

\subsection{規制遵守領域における相違}

\begin{table}[h]
\centering
\caption{意思決定木アプローチの比較}
\begin{tabular}{@{}lccc@{}}
\toprule
\textbf{観点} & \textbf{ID3/C4.5} & \textbf{CART} & \textbf{本稿} \\
\midrule
分割基準 & IG / Gain Ratio & Gini & IG + 終端確率 \\
木構造 & 多分岐 & 二分 & 多分岐 \\
多値対応 & 可 & 不可 & 可 \\
法的制約 & 考慮しない & 考慮しない & 考慮する \\
早期終了 & 暗黙 & 暗黙 & 明示 \\
質問統合 & なし & なし & あり \\
\bottomrule
\end{tabular}
\end{table}

規制遵守では \textbf{正確性が必須} であり,効率のために法的結論を犠牲にできない点が最大の違いである.

\section{最適化手法}

\subsection{早期終了(枝刈り)}

終端に到達しやすい質問は木の早い段階に配置すべきである.

\begin{definition}[Early Termination Score]
質問 $q$ の終端確率 $p_{\text{term}}$ に対して
\begin{equation}
\text{score}(q) = \text{IG}(q) \times (1 + \alpha \cdot p_{\text{term}})
\end{equation}
と定義する.$\alpha$ は終端確率の重み(通常 0.5--1.0)である.
\end{definition}

\subsection{質問統合}

同様の情報構造を持つ質問を統合できる.

\begin{definition}[Mergeability Criterion]
質問 $q_1, q_2, \ldots, q_k$ は以下を満たす場合に統合可能とする.
\begin{enumerate}
    \item 回答集合が相互排他的:$\forall i,j: A(q_i) \cap A(q_j) = \emptyset$
    \item 意味的類似度が閾値を上回る
    \item 認知負荷が許容範囲
\end{enumerate}
\end{definition}

\begin{theorem}[Information Preservation]
統合後の質問 $q_m$ について
\begin{equation}
\text{IG}(q_m) \geq \max(\text{IG}(q_1), \ldots, \text{IG}(q_k))
\end{equation}
\end{theorem}

\subsection{遅延評価(Lazy Evaluation)}

先行回答から推論できる質問は省略する.

\begin{example}[Inference Rules]
\begin{align*}
&\text{flag\_is\_provider} = \text{TRUE} \Rightarrow \text{skip Q8 (modifications)} \\
&\neg(\text{flag\_high\_risk} \land \text{flag\_is\_deployer}) \Rightarrow \text{skip Q9 (public body)}
\end{align*}
\end{example}

\section{ケーススタディ:EU AI Act 適合性チェッカー}

\subsection{データセット}

\begin{table}[h]
\centering
\caption{質問票の統計}
\begin{tabular}{@{}lr@{}}
\toprule
\textbf{指標} & \textbf{値} \\
\midrule
総質問数 & 33 \\
QAIS(AI System)質問 & 27 \\
QGPAI(GPAI Model)質問 & 5 \\
その他(導入) & 1 \\
終端状態(END) & 20 \\
列挙パス数 & 7,762 \\
\bottomrule
\end{tabular}
\end{table}

\subsection{パス長の分析}

\begin{table}[h]
\centering
\caption{パス長の統計}
\begin{tabular}{@{}lr@{}}
\toprule
\textbf{指標} & \textbf{質問数} \\
\midrule
最短 & 2 \\
最長 & 12 \\
平均 & 8.60 \\
中央値 & 8 \\
\bottomrule
\end{tabular}
\end{table}

\subsection{情報利得の分析}

\begin{table}[h]
\centering
\caption{情報利得上位の質問}
\begin{tabular}{@{}lrrr@{}}
\toprule
\textbf{質問} & \textbf{IG} & \textbf{Paths} & \textbf{終端確率} \\
\midrule
QAIS 6.3 & 4.6509 & 392 & 0.0\% \\
QGPAI 3.1 & 4.6497 & 7 & 71.4\% \\
QGPAI 7 & 4.6496 & 6 & 100.0\% \\
QGPAI 3 & 4.6469 & 18 & 38.9\% \\
QGPAI 2 & 4.6455 & 18 & 0.0\% \\
QGPAI 1 & 4.6432 & 19 & 5.3\% \\
QAIS 2.2 & 4.3986 & 1,029 & 0.7\% \\
QAIS 7.1 & 4.1793 & 1,848 & 100.0\% \\
\bottomrule
\end{tabular}
\end{table}

\subsection{低情報利得の質問}

\begin{table}[h]
\centering
\caption{低情報利得の質問}
\begin{tabular}{@{}lrr@{}}
\toprule
\textbf{質問} & \textbf{IG} & \textbf{Paths} \\
\midrule
Q1(導入) & 0.0248 & 7,762 \\
QAIS 1 & 0.0349 & 7,743 \\
QAIS 3 & 0.1646 & 7,728 \\
\bottomrule
\end{tabular}
\end{table}

\subsection{早期終了ポイント}

\begin{table}[h]
\centering
\caption{早期終了候補}
\begin{tabular}{@{}lrr@{}}
\toprule
\textbf{質問} & \textbf{終端確率} & \textbf{平均深さ} \\
\midrule
QAIS 5(禁止) & 61.5\% & 7.7 \\
QAIS 7.1(高リスク) & 100.0\% & 11.9 \\
QGPAI 3(システミックリスク) & 38.9\% & 4.0 \\
QGPAI 3.1(オープンソース) & 71.4\% & 5.0 \\
QGPAI 7(最終) & 100.0\% & 5.3 \\
\bottomrule
\end{tabular}
\end{table}

\textbf{示唆}:QAIS 5 は終端確率 61.5\% と高い一方で平均深さ 7.7 に配置されている.これを早い段階に移動できれば,多数のパスを短縮できる.

\subsection{最適化提案}

\begin{enumerate}
    \item \textbf{QAIS 5 の前倒し}:禁止慣行チェックをより早い深さ(例:4--5)へ
    \item \textbf{QAIS 6.4.1--6.4.7 の統合}:Annex III の7つの小問を1問に統合
    \item \textbf{Q1 と QAIS 1 の統合}:導入質問とAIシステム定義の同時確認
    \item \textbf{GPAI トラックの維持}:もともと平均4問程度で効率的
\end{enumerate}

\section{結果}

\subsection{定量的改善}

\begin{table}[h]
\centering
\caption{最適化結果}
\begin{tabular}{@{}lrrr@{}}
\toprule
\textbf{指標} & \textbf{元} & \textbf{最適化後} & \textbf{改善} \\
\midrule
総質問数 & 33 & 20 & 39\%削減 \\
平均パス(AI System) & 8.6 & 6--7 & 20--30\%削減 \\
平均パス(GPAI) & 4--5 & 3--4 & 20--25\%削減 \\
終端状態 & 20 & 12 & 40\%削減 \\
\bottomrule
\end{tabular}
\end{table}

\subsection{情報理論的効率}

\begin{equation}
\text{Question Efficiency Score (QES)} = \frac{\sum \text{IG}(q_i)}{|\text{questions asked}|}
\end{equation}

\begin{align}
\text{Original QES} &= \frac{7.8}{8.6} = 0.91 \\
\text{Optimized QES} &= \frac{7.8}{6.0} = 1.30 \\
\text{Improvement} &= 43\%
\end{align}

\section{実装}

\subsection{最適化エンジンの構成}

Python により最適化エンジンを実装した(要旨).

\begin{lstlisting}[language=Python]
class ECRoutingAnalyzer:
    def analyze(self) -> Dict[str, Any]:
        # Enumerate all paths
        self._enumerate_paths('Q1', [], [], {})
        # Calculate statistics
        self._calculate_stats()
        return self._generate_report()

    def _calculate_information_gain(self):
        # Shannon entropy calculation
        overall_entropy = self._entropy(probabilities)
        # Conditional entropy
        for qid in self.question_stats:
            partitions = self._partition_by_answer(qid)
            conditional_entropy = sum(
                prob * self._entropy(partition)
                for prob, partition in partitions
            )
            ig = overall_entropy - conditional_entropy
\end{lstlisting}

\subsection{検証}

以下により正確性を確認した.

\begin{enumerate}
    \item \textbf{パス列挙}:7,762本の全パスが正しく終端すること
    \item \textbf{法的整合}:各結論がEU AI Actの要件と一致すること
    \item \textbf{同値性テスト}:最適化フローが元フローと同一結論を返すこと
\end{enumerate}

\section{考察}

\subsection{C4.5との比較}

C4.5 と本稿はいずれも情報利得に基づくが,(i) 法的制約がハード制約であること,(ii) 終端確率を明示的に重み付けすること,(iii) 質問統合にドメイン知識が必要なこと,(iv) AI System と GPAI の多トラック構造を扱うこと,が異なる.

\subsection{限界}

\begin{enumerate}
    \item \textbf{パス確率推定}:回答分布を一様と仮定した
    \item \textbf{ユーザ行動}:実ユーザは最適パスを辿らない可能性
    \item \textbf{法改正}:AI Actの更新により再最適化が必要
\end{enumerate}

\subsection{汎用性}

本枠組みは GDPR,医療機器分類(MDR),金融規制(MiFID II),環境規制(REACH)などにも適用可能である.

\section{結論}

本稿は,規制遵守の意思決定木に対し,情報理論に基づいて平均質問数を削減する枠組みを示した.EU AI Act 適合性チェッカーに適用した結果,総質問数の削減(33 $\rightarrow$ 20)および平均パス長の 20--30\% 程度の改善を得た.

\section*{謝辞}

データは European Commission AI Act Service Desk のコンプライアンスチェッカーに基づく.

\begin{thebibliography}{9}

\bibitem{quinlan1986}
Quinlan, J.R. (1986). Induction of decision trees. \textit{Machine Learning}, 1(1), 81--106.

\bibitem{quinlan1993}
Quinlan, J.R. (1993). \textit{C4.5: Programs for Machine Learning}. Morgan Kaufmann.

\bibitem{breiman1984}
Breiman, L., Friedman, J., Olshen, R., \& Stone, C. (1984). \textit{Classification and Regression Trees}. Wadsworth.

\bibitem{shannon1948}
Shannon, C.E. (1948). A mathematical theory of communication. \textit{Bell System Technical Journal}, 27(3), 379--423.

\bibitem{euaiact2024}
European Parliament and Council (2024). Regulation (EU) 2024/1689 (AI Act). \textit{Official Journal of the European Union}.

\end{thebibliography}

\appendix

\section{質問統計}

\begin{table}[h]
\centering
\small
\caption{最適化エンジンによる質問統計(抜粋)}
\begin{tabular}{@{}lrrrr@{}}
\toprule
\textbf{質問} & \textbf{Paths} & \textbf{Avg Depth} & \textbf{終端確率} & \textbf{IG} \\
\midrule
Q1 & 7,762 & 1.00 & 0.0\% & 0.0248 \\
QAIS 1 & 7,743 & 2.00 & 0.0\% & 0.0349 \\
QAIS 1.1 & 6,636 & 3.00 & 0.0\% & 0.6944 \\
QAIS 2 & 7,742 & 3.86 & 0.1\% & 2.4332 \\
QAIS 2.1 & 5,628 & 4.86 & 0.0\% & 2.2282 \\
QAIS 2.2 & 1,029 & 4.86 & 0.7\% & 4.3986 \\
QAIS 3 & 7,728 & 5.72 & 1.4\% & 0.1646 \\
QAIS 4 & 7,616 & 6.72 & 23.5\% & 0.9800 \\
QAIS 5 & 5,824 & 7.71 & 61.5\% & 2.3900 \\
QAIS 6 & 2,240 & 8.68 & 0.0\% & 3.7676 \\
QAIS 6.1 & 2,240 & 9.68 & 0.0\% & 3.7676 \\
QAIS 6.2 & 2,240 & 10.86 & 17.5\% & 3.7676 \\
QAIS 6.3 & 392 & 9.86 & 0.0\% & 4.6509 \\
QAIS 7.1 & 1,848 & 11.86 & 100.0\% & 4.1793 \\
\midrule
QGPAI 1 & 19 & 2.00 & 5.3\% & 4.6432 \\
QGPAI 2 & 18 &  3.00 & 0.0\% & 4.6455 \\
QGPAI 3 & 18 & 4.00 & 38.9\% & 4.6469 \\
QGPAI 3.1 & 7 & 5.00 & 71.4\% & 4.6497 \\
QGPAI 7 & 6 & 5.33 & 100.0\% & 4.6496 \\
\bottomrule
\end{tabular}
\end{table}

\end{document}